Automated disease detection in videos and images from the gastrointestinal tract has received much attention in the last years.
However, the quality of image data is often reduced due to overlays of text, personal data, and black corners around the medical images.

Machine learning has become a prevalent tool to help with medical diagnosis and notably supervised machine learning has shown great success recently.
Though a problem often encountered when using machine learning for medical diagnosis is the fact that the anomalies found in images, like the text and rounded corners, results in higher misclassification.


In this thesis, we tackle the problems associated with the misclassification of data based on overlays and other artefacts in the medical image data.
We will go in-depth into the topic of preprocessing to see if it has a place in modern classification models based on deep learning. 

During this thesis, we will look at different tools that we can use to remove dataset specific artefacts, and we will look at the consequences of removing them.
Our primary focus lies in the usage of generative adversarial neural networks to cover up parts of images that we have deemed unwanted in our medical images.